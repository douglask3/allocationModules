\section{Allometric}
\label{Allometric}




\begin{equation}
	\frac{A_{root}=A_{root,max}*A_{root,min}}{A_{root,min}+(A_{root,max}-A_{root,min})*L}
\end{equation}

Tree height ($H$) based on an allometric relationship described in \citet{Causton1985}:

\begin{equation}
	H = H_{t} * c_{stem}^{h_{t}}
\end{equation}
where $c_{stem}$ is the circumerfance of the stem and $H_{t}$ \& $h_{t}$ are allomtric paramters.
                                
Leaf Area Index ($LAI$) to stem sapwood cross-sectional area ($\oslash_{sapwood}$) is assumed to vary between $\oslash_{sapwood,0}$ and $\oslash_{sapwood,1}$ as a linear function of $H$ 

\begin{equation}
	LS = LAI / \oslash_{sapwood}
\end{equation}

where $\oslash_{sapwood}$ is calculated as:

\begin{equation}
	\oslash_{sapwood} = C_{sapwood}/(H * P *C_{dry} )
\end{equation}
where $C_{sapwood}$ is the total carbon content of sapwood, $P$ is population density and $C_{dry}$ is the carbon fraction of dry mass.

Allocation to leaves dependant on height. Modification of pipe theory, leaf-to-sapwood ratio is not constant above a certain  height, due to hydraulic constraints \citep{Magnani2000,Deckmun2006}

\begin{equation}
	LS_{target}=
	\begin{cases}
		    LS_{0},& \text{if } H\le H_{0} \\ 
		    LS_{1},& \text{if } H\ge H_{1} \\
		    LS_{0} + \dfrac{(LS_{1}-LS_{0})\cdot (H-H_{0})}{H_{1}-H_{0}},& \text{otherwise}
		\end{cases}
\end{equation}

Sensitivity parameter ($A_{s}$) characterises how allocation fraction respond when the leaf:sapwood area ratio departs from the target value. If $A_{s}$ is close to 0 then the simulated leaf:sapwood area ratio will closely track the target value

\begin{equation}
	A_{leaf} = 0.5 + 0.5 \cdot \dfrac{(1-LS/LS_{target})}{A_{s}}
	\label{AleafAllometric}
\end{equation}

root allocation depends on available water \& nutrients (as defined by equations \ref{e:qlim1} -- \ref{eq:limn} and is adjusted below as we aim to maintain a functional balance.

\begin{equation}
	A_{root} = \dfrac{A_{root,max} \cdot A_{root,min}}{A_{root,min} + (A_{root,max} A_{root,min})*L}
\end{equation}

calculate imbalance, based on ``biomass'' as described by \citet{sitch2003}

\begin{equation}
	\Delta_{miss}=
	\begin{cases}
	    1.9,& \text{if } root mass = 0 \\
	    C_{shoot}/(C_{shoot} \cdot L),& \text{otherwise}
	\end{cases}
\end{equation}

If $\Delta_{miss} > 1.0$ 
\begin{equation}
	A_{leaf} = \max (A_{leaf,min}, \min (A_{leaf,max}, A_{leaf}^* / \Delta_{miss}))
\end{equation}
\begin{equation}
	A_{root} = A_{root}^* + \max (A_{root,min}, A_{leaf}^* - A_{leaf})
\end{equation}

If $\Delta_{miss} < 1.0$ 
\begin{equation}
	A_{root} = \max (A_{root,min}, \min (A_{root,max},A_{root,max}, A_{root}^* \cdot \Delta_{miss}))
\end{equation}
\begin{equation}
	A_{leaf} = A_{leaf}^* + \max (A_{leaf,max}, \max(0.0, A_{root}^* - A_{root}))
\end{equation}


Allocation to branch dependent on relationship between the stem and branch
\begin{equation}
	C_{target,branch} = branch0 \cdot C_{stem}^branch0
\end{equation}

\begin{equation}
	A_{branch} = 0.5 + 0.5 \cdot \dfrac{(1-C_{branch}/C_{target,branch})}{A_{s}}
\end{equation}

